\documentclass[a4paper]{article}
\usepackage{listings}
\usepackage{amsmath}
\usepackage{amsfonts}
% vim: ts=2 sw=2 noet

\newcommand{\F}[1]{\mathrm{GF} ({#1})}
% enclose element in braces.
\newcommand{\be}[1]{\{{#1}\}}
\makeatletter
\lstnewenvironment{c++}[1][]{%
	\lstset{%
		#1,
		language=c++,
		numbers=left,
		firstnumber=auto,
		tabsize=1,
		basicstyle=\small\ttfamily,
		mathescape=true,
	}%
		\csname\@lst @SetFirstNumber\endcsname
	}{%
		\csname \@lst @SaveFirstNumber\endcsname
	}
\makeatother

\title{Commentary on the Bech32 Polynomial Generator Function}
\date{July 2017}
\author{Pieter Wuille}
\begin{document}
\maketitle
This function will compute what six five-bit values to XOR ($\oplus$) into the
last six input values, in order to make the checksum 0.
These six values are packed together in a single 30-bit integer.
The higher bits correspond to earlier values.
\begin{c++}[name=Code]
uint32_t polymod(const data& values)
{
\end{c++}
The input is interpreted as a list of coefficients of a polynomial over
$F = \F{32}$, with an implicit 1 in front. If the input is
$[v_0, v_1, v_2, v_3, v_4]$, that polynomial is
\begin{equation*}
	v(x) = 1x^5 + v_0x^4 + v_1x^3 + v_2x^2 + v_3x + v_4.
\end{equation*}
The implicit 1 guarantees that $[v_0, v_1, v_2, \ldots]$ has a distinct checksum
from $[0, v_0, v_1, v_2, \ldots]$.

The output is a 30-bit integer whose 5-bit groups are the coefficients of the
remainder of $v(x) \mod g(x)$, where $g(x)$ is the Bech32 generator:
\begin{equation}
  g(x) = x^6 + 29x^5 + 22x^4 + 20x^3 + 21x^2 + 29x + 18.
\end{equation}
$g(x)$ is chosen in such a way that the resulting code is a BCH code,
guaranteeing detection of up to three errors within a window of 1023 characters.
Among the various possible BCH codes, one was selected to in fact guarantee
detection of up to 4 errors within a window of 89 characters.

Note that the coefficients are elements of $\F{32}$, here represented as decimal
numbers between $\{\}$.
In this finite field, addition is just XOR of the corresponding numbers.
For example, $\be{27} + \be{13} = \be{27} \oplus \be{13} = \be{22}$.
Multiplication is more complicated, and requires treating the bits of values
themselves as coefficients of a polynomial over a smaller field, $\F{2}$, and
multiplying those polynomials $\mod a^5 + a^3 + 1$. 
For example:
\begin{align*}
	5 \cdot 26	&= (a^2 + 1) \cdot (a^4 + a^3 + a) \\
			&= (a^4 + a^3 + a) \cdot a^2 + (a^4 + a^3 + a) \\
			&= a^6 + a^5 + a^4 + a \\
			&= a^3 + 1 \pmod{a^5 + a^3 + 1} \\
      &= \be{9}.
\end{align*}

During the course of the loop below, \texttt{c} contains the bitpacked coefficients of
the polynomial constructed from just the values of $v$ that were processed so far,
$\mod g(x)$.
In the above example, \texttt{c} initially corresponds to $1 \mod g(x)$, and
after processing 2 inputs of $v$, it corresponds to $x^2 + v_0x + v_1 \mod g(x)$.
As $1 \mod g(x) = 1$, that is the starting value for \texttt{c}.
\begin{c++}[name=Code]
	uint32_t c = 1;
	for (const auto $v_i$ : values) {
\end{c++}
We want to update \texttt{c} to correspond to a polynomial with one extra term.
If the initial value of \texttt{c} consists of the coefficients of
$c(x) = f(x) \mod g(x)$, we modify it to correspond to
$c'(x) = (f(x) \cdot x + v_i) \mod g(x)$, where $v_i$ is the next input to process.
Simplifying:
\begin{equation}
\begin{aligned}
	c'(x)	&= (f(x) \cdot x + v_i) \mod g(x) \\
		&= ((f(x) \mod g(x)) \cdot x + v_i) \mod g(x) \\
		&= (c(x) \cdot x  + v_i) \mod g(x)
\end{aligned}
\end{equation}
If
\begin{equation*}
	c(x) = c_0x^5 + c_1x^4 + c_2x^3 + c_3x^2 + c_4x + c_5, 
\end{equation*}
we want to compute:
\begin{equation*}
\begin{aligned}
	c'(x)	&= (c_0x^5 + c_1x^4 + c_2x^3 + c_3x^2 + c_4x + c_5) \cdot x + v_i \mod g(x) \\
		&= c_0x^6 + c_1x^5 + c_2x^4 + c_3x^3 + c_4x^2 + c_5x + v_i \mod g(x) \\
		&= c_0(x^6 \mod g(x)) + c_1x^5 + c_2x^4 + c_3x^3 + c_4x^2 + c_5x + v_i
\end{aligned}
\end{equation*}
If we call $(x^6 \mod g(x)) = k(x)$, this can be written as:
\begin{equation*}
	c'(x)	= (c_1x^5 + c_2x^4 + c_3x^3 + c_4x^2 + c_5x + v_i) + c_0k(x)
\end{equation*}

First, determine the value of $c_0$:
\begin{c++}[name=Code]
		uint8_t $c_0$ = c >> 25;
\end{c++}
Then compute $c_1x^5 + c_2x^4 + c_3x^3 + c_4x^2 + c_5x + v_i$:
\begin{c++}[name=Code]
		c = ((c & 0x1FFFFFF) << 5) ^ $v_i$;
\end{c++}
Finally, for each set bit $n$ in $c_0$, conditionally add ${2^n}k(x)$:
\begin{equation*}
\begin{aligned}
       k(x) &= \be{29}x^5 + \be{22}x^4 + \be{20}x^3 + \be{21}x^2 + \be{29}x + \be{18} \\
 \be{2}k(x) &= \be{19}x^5 +  \be{5}x^4 +        x^3 +  \be{3}x^2 + \be{19}x + \be{13} \\
 \be{4}k(x) &= \be{15}x^5 + \be{10}x^4 +  \be{2}x^3 +  \be{6}x^2 + \be{15}x + \be{26} \\ 
 \be{8}k(x) &= \be{30}x^5 + \be{20}x^4 +  \be{4}x^3 + \be{12}x^2 + \be{30}x + \be{29} \\
\be{16}k(x) &= \be{21}x^5 +        x^4 +  \be{8}x^3 + \be{24}x^2 + \be{21}x + \be{19} \\
\end{aligned}
\end{equation*}
\begin{c++}[name=Code]
		if ($c_0$ & 1 )  c ^= 0x3b6a57b2;
		if ($c_0$ & 2 )  c ^= 0x26508e6d;
		if ($c_0$ & 4 )  c ^= 0x1ea119fa;
		if ($c_0$ & 8 )  c ^= 0x3d4233dd;
		if ($c_0$ & 16)  c ^= 0x2a1462b3;
	}
	return c;
}
\end{c++}
\end{document}
